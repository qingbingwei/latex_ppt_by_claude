%!TEX program = xelatex
\documentclass[aspectratio=169,11pt]{beamer}

% 中文支持
\usepackage[UTF8]{ctex}

% 主题设置
\usetheme{Madrid}
\usecolortheme{dolphin}

% 常用宏包
\usepackage{amsmath, amssymb, amsfonts}
\usepackage{graphicx}
\usepackage{hyperref}
\usepackage{booktabs}
\usepackage{ulem}

% 元信息
\title{中国人民抗日战争}
\subtitle{历史回顾与伟大胜利}
\author{GitHub Copilot}
\institute{AI 助手}
\date{\today}

\begin{document}

% 标题页
\begin{frame}
  \titlepage
\end{frame}

% 目录页
\begin{frame}{目录}
  \tableofcontents
\end{frame}

\section{战争背景与爆发}

\begin{frame}{九一八事变:局部抗战开始}
  \begin{itemize}
    \item \textbf{时间}:1931年9月18日
    \item \textbf{地点}:沈阳柳条湖
    \item \textbf{经过}:日本关东军炸毁南满铁路,反诬中国军队破坏,炮轰北大营。
    \item \textbf{后果}:东北三省迅速沦陷,中国人民开始局部抗战。
  \end{itemize}
\end{frame}

\begin{frame}{七七事变:全民族抗战爆发}
  \begin{itemize}
    \item \textbf{时间}:1937年7月7日
    \item \textbf{地点}:北平卢沟桥
    \item \textbf{标志}:日本发动全面侵华战争,中国全民族抗战由此开始。
    \item \textbf{口号}:“平津危急!华北危急!中华民族危急!”
  \end{itemize}
\end{frame}

\section{主要战役与战场}

\begin{frame}{战略防御阶段:正面战场}
  \begin{itemize}
    \item \textbf{淞沪会战}(1937.8):打破了日军“三个月灭亡中国”的迷梦。
    \item \textbf{南京保卫战}(1937.12):南京失守,日军制造了惨绝人寰的南京大屠杀。
    \item \textbf{台儿庄战役}(1938.3):抗战以来中国军队取得的重大胜利。
  \end{itemize}
\end{frame}

\begin{frame}{敌后战场的中流砥柱}
  \begin{itemize}
    \item \textbf{平型关大捷}(1937.9):全民族抗战以来中国军队的第一个大胜仗。
    \item \textbf{百团大战}(1940.8):八路军在华北发动的大规模进攻作战。
    \item \textbf{游击战争}:地道战、地雷战等灵活战术,陷入人民战争的汪洋大海。
  \end{itemize}
\end{frame}

\section{抗日民族统一战线}

\begin{frame}{国共第二次合作}
  \begin{itemize}
    \item \textbf{西安事变}(1936.12):和平解决,标志着十年内战基本结束,抗日民族统一战线初步形成。
    \item \textbf{正式形成}(1937.9):国民党公布国共合作宣言。
    \item \textbf{意义}:团结一切可以团结的力量,共同抵抗日本侵略者。
  \end{itemize}
\end{frame}

\section{伟大胜利与历史意义}

\begin{frame}{日本无条件投降}
  \begin{itemize}
    \item \textbf{1945年8月15日}:日本天皇裕仁广播《终战诏书》,宣布无条件投降。
    \item \textbf{1945年9月2日}:日本代表在“密苏里号”战列舰上签署投降书。
    \item \textbf{1945年9月9日}:中国战区受降仪式在南京举行。
  \end{itemize}
\end{frame}

\begin{frame}{抗战胜利的伟大意义}
  \begin{itemize}
    \item 近代以来中国抗击外敌入侵的第一次完全胜利。
    \item 促进了中华民族的觉醒,为中国共产党带领中国人民实现彻底独立奠定了基础。
    \item 世界反法西斯战争的重要组成部分,为世界和平做出了巨大贡献。
    \item 重新确立了中国在世界上的大国地位。
  \end{itemize}
\end{frame}

\section{结语}

\begin{frame}{铭记历史 珍爱和平}
  \begin{center}
    \Huge 铭记历史 \\ 缅怀先烈 \\ 珍爱和平 \\ 开创未来
  \end{center}
\end{frame}

\end{document}
